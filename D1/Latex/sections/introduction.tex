\chapter{Introduction}

\section{Objective}
Descriptive statistics play a vital role in measurably condensing data, providing valuable understanding into data sets’ central tendencies, variability, and frequency distributions. \mbox{METRICSTICS}, a blend word of "METRICS" and "STATISTICS", represents a system designed for the \mbox{computation} of key descriptive statistics, including minimum $(m)$, maximum $(M)$, arithmetic mean $(\mu)$, mean absolute deviation (MAD), and standard deviation $(\sigma)$. This system is intended to intake a \mbox{variable} number of data values and produce their respective descriptive statistics. The data could be real-world, gained from comprehensive sources, or simulated with the help of random data producers.
\newline

\noindent The current extent of the project is limited to:
\begin{enumerate}
    \item An input of arbitrary values from confined data set $(N_1, N_2, N_3, \ldots,N_{M}).$
    \item An output of any one of the following:
    \begin{itemize}
        \item Descriptive Statistics: METRICSTICS produces statistics like minimum $(m)$, \mbox{maximum$(M)$}, mean $(\mu)$, mean absolute deviation (MAD), and standard deviation $(\sigma)$.
        \item Error Handling: The system must clearly define the error messages when it is \mbox{incompetent} in processing the data.
    \end{itemize}
\end{enumerate}