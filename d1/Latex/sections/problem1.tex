\chapter{Problem 1}

\section{Problem Statement}

Team needs to specify a goal using the \ac{GQM} approach and formulate couple of questions per team member to reach at the defined goal within the given time span and figure out if there are any metrics that help us to answer those questions so that the required goal can be achieved.

\section{Goal Formulation}

The SMART goal for METRICSTICS is to develop an \textbf{efficient} and \textbf{accurate} system for \mbox{calculating} descriptive statistics from the given finite data sets within the upcoming four-month period.

\section{SMART}

SMART \cite{smart-principle} principle consists of:
\begin{itemize}
\item \textbf{Specific}: The goal noticeably states what needs to be accomplished, which is the development \mbox{METRICSTICS} for descriptive statistics.
\item \textbf{Measurable}: The objective can be measured by evaluating the efficiency and accuracy of the system.
\item \textbf{Achievable}: Considering the project description, it is attainable to develop such a system.
\item \textbf{Relevant}: The goal is directly related to project's  purpose.
\item \textbf{Time-bound}: A time frame of four months is specified for achieving the goal.
\end{itemize}

\section{Questions Articulation \& Metric Discussion}
We articulate the following questions to guide the project:
\begin{enumerate}
    \item How can we ensure the accuracy of descriptive statistics calculations in METRICSTICS?

    \textbf{Metric}: \textbf{Mean Absolute Error (MAE)} \cite{mean-absolute-error} - Calculate the MAE between the \mbox{METRICSTICS'} descriptive statistics that has been computed and the actual value (known) of the sample data sets provided as input. The lower the value of MAE, better is the correctness of the system.

    \item What characteristics or capabilities should be prioritized in METRICSTICS development?

    \textbf{Metric}: \textbf{User Satisfaction Score} \cite{customer-satisfaction-score} - Gather feedback from users on the importance and satisfaction level of various features and functionalities present in the system by conducting surveys and analyzing the reviews section of the system. Prioritize those functionalities that have higher user satisfaction scores as they are most likely to be reused by users.


    \item How can METRICSTICS enhance the performance of data sets with huge amount of data?
    
    \textbf{Metric}: \textbf{Processing Time} - Measure the time taken by METRICSTICS to calculate \mbox{descriptive} statistics for many data sets of various sizes. More optimizations, for instance using optimized algorithms or formulas, leads to less processing time.

    \item What user requirements or expectations need consideration in METRICSTICS development?

    \textbf{Metric}: \textbf{User Requirement Compliance} - Analyze the extent to which METRICSTICS meets upon the agreed user requirements. A compliance percentage examines the degree to which system aligns with user expectations from the system.

    \item What strategies can enhance the user interface and experience?

    \textbf{Metric}: \textbf{User Interface Usability Score} \cite{user-usability} - Conduct usability testing with users to evaluate how easy and intuitive it is for users to navigate or access the system using the given user interface. A higher usability score indicates a better user experience.

    \item How will METRICSTICS handle missing or incomplete data?

    \textbf{Metric}: \textbf{Handling Efficiency} - Evaluate the system's effectiveness in handling missing or incomplete data by analyzing difference in processing time(key metric) between complete and missing data and resource utilization(key metric) like CPU usage, RAM(memory) usage and disk usage.

    \item What are potential challenges or risks in METRICSTICS development?

    \textbf{Metric}: \textbf{Risk Severity Index} \cite{risk-severity} - Assign a severity index to the risks that have been \mbox{recognized} based on the impact they can make on project goals or deliverable like exceeding project budget or crossing the project deadline. Put considerable efforts to reduce the impact of high-severity risks.

    \item How can METRICSTICS ensure compliance with data privacy and security regulations?

    \textbf{Metric}: \textbf{Compliance Percentage} \cite{compliance-audit} - Conduct regular compliance audits to make sure that the \mbox{METRICSTICS} adheres to data privacy and security regulations. Compliance percentage indicates how much system is compliant with standards, rules, and laws.

    \item What is the estimated development timeline, and how can it be maintained?

    \textbf{Metric}: \textbf{Task Completion Progress} \cite{task-completion}- Track the percentage of completed development tasks against the project schedule. Monitor adherence to upcoming tasks and milestones to make sure that the timeline is maintained and there are less high-severity risks.

    \item How will the team coordinate effectively?

    \textbf{Metric}: \textbf{Task Collaboration Efficiency} - Assess the efficiency of team coordination by \mbox{measuring} the time taken to complete collaborative tasks and the frequency of successful task \mbox{completions}.
\end{enumerate}

\noindent These metrics provide quantifiable measures for evaluating progress and performance related to each of the articulated questions, ensuring that project aims are met effectively.